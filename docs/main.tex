\documentclass[a4paper,12pt]{article}
\usepackage[utf8]{inputenc}
\usepackage[russian]{babel}
\usepackage{amsmath}
\usepackage{listings} % For code highlighting
\usepackage{xcolor} % For syntax highlighting colors
\usepackage{hyperref} % For clickable links

% Code Syntax Highlighting Settings
\lstset{
    language=Python,
    basicstyle=\ttfamily\small,
    keywordstyle=\color{blue},
    commentstyle=\color{gray},
    stringstyle=\color{red},
    showstringspaces=false,
    numbers=left,
    numberstyle=\tiny\color{gray},
    stepnumber=1,
    numbersep=10pt,
    backgroundcolor=\color{white},
    tabsize=4,
    breaklines=true,
    breakatwhitespace=false,
    frame=single,
    captionpos=b
}

\begin{document}

% Title Page
\begin{titlepage}
    \centering
    {\large \textbf{Лабораторная работа}}\\[1cm]
    {\Large \textbf{Документация к библиотеке geometric\_lib}}\\[2cm]
    \textbf{Выполнил:} МД Абу Хуссаин Рифад \\[0.5cm]
    \textbf{Группа:} M3113 \\[0.5cm]
    \textbf{Преподаватель:} Жуйков Артём Сергеевич \\[2cm]
    \vfill
    Санкт-Петербург, 2024 г.
\end{titlepage}

% Table of Contents
\tableofcontents
\newpage

% General Description of the Library
\section{Общее описание библиотеки}
Библиотека \texttt{geometric\_lib} предназначена для вычисления площадей и периметров различных геометрических фигур, таких как круг, прямоугольник, квадрат и треугольник. 

Репозиторий библиотеки доступен по ссылке:  
\url{https://github.com/smartiqaorg/geometric_lib}

\newpage

% Code and Logic Description
\section{Описание файлов программ из репозитория}

\subsection*{Круг (Circle)}
\textbf{Функции:}
\begin{itemize}
    \item \texttt{area(r)}: вычисляет площадь круга по радиусу \( r \) по формуле:  
    \[
    S = \pi r^2
    \]
    \item \texttt{perimeter(r)}: вычисляет длину окружности по формуле:  
    \[
    P = 2 \pi r
    \]
\end{itemize}

\begin{lstlisting}[caption={Код для вычисления площади и периметра круга}]
import math

def area(r):
    return math.pi * r * r

def perimeter(r):
    return 2 * math.pi * r
\end{lstlisting}

\newpage

\subsection*{Прямоугольник (Rectangle)}
\textbf{Функции:}
\begin{itemize}
    \item \texttt{area(a, b)}: вычисляет площадь прямоугольника с длиной \( a \) и шириной \( b \):  
    \[
    S = a \cdot b
    \]
    \item \texttt{perimeter(a, b)}: вычисляет периметр прямоугольника:  
    \[
    P = 2(a + b)
    \]
\end{itemize}

\begin{lstlisting}[caption={Код для вычисления площади и периметра прямоугольника}]
def area(a, b):
    return a * b

def perimeter(a, b):
    return 2 * (a + b)
\end{lstlisting}

\newpage

\subsection*{Квадрат (Square)}
\textbf{Функции:}
\begin{itemize}
    \item \texttt{area(a)}: вычисляет площадь квадрата со стороной \( a \):  
    \[
    S = a^2
    \]
    \item \texttt{perimeter(a)}: вычисляет периметр квадрата:  
    \[
    P = 4a
    \]
\end{itemize}

\begin{lstlisting}[caption={Код для вычисления площади и периметра квадрата}]
def area(a):
    return a * a

def perimeter(a):
    return 4 * a
\end{lstlisting}

\newpage

\subsection*{Треугольник (Triangle)}
\textbf{Функции:}
\begin{itemize}
    \item \texttt{area(a, h)}: вычисляет площадь треугольника с основанием \( a \) и высотой \( h \):  
    \[
    S = \frac{a \cdot h}{2}
    \]
    \item \texttt{perimeter(a, b, c)}: вычисляет периметр треугольника как сумму сторон:  
    \[
    P = a + b + c
    \]
\end{itemize}

\begin{lstlisting}[caption={Код для вычисления площади и периметра треугольника}]
def area(a, h):
    return a * h / 2

def perimeter(a, b, c):
    return a + b + c
\end{lstlisting}

\newpage

% Links to the Project
\section{Ссылки на проект}
По следующим ссылкам можно просмотреть исходный код и документацию проекта:

\begin{itemize}
    \item \textbf{GitHub репозиторий:}  
    \url{https://github.com/smartiqaorg/geometric_lib}

    \item \textbf{Overleaf проект:}  
    \url{https://www.overleaf.com/read/rmdhyfxshfrt#796a76}
\end{itemize}

\end{document}